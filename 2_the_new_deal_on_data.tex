\section{Essential Elements of the New Deal on Data in the Context of Institutional Controls (Arek)}

NOTE: I write 'we' everywhere, but feel free to change if some work should be attributed more precisely.

To realize the promise and prospects of Big Data, and to avoid its privacy perils, we need a balanced set of institutional controls.
Theses controls must support and reflect a greater user control over personal data, as well as large scale interoperability for data sharing between and among institutions.
The core capabilities of theses controls should include responsive rule-based systems governance and fine grained authorizations for distributed rights management.

Our lives are embedded within institutions. 
We are citizens of the countries and cities, receive services from telecom operators, search for things to buy in the online stores. 
All the actions we perform generate data, and those breadcrumbs of our lives are an important part of the Big Data promise.
The data that are not curated by us, but are collected as is, reflecting our lives.

Today, all these data that people generate in the context of institutions, are closed in silos. 
The mobility traces are owned by the phone providers, music tastes are stored and used by music services.
For these data to be useful for society, they must be opened, must be used much more than they are today.
If the access to the data for the purpose of creating the value, either for the user or for the society, is very limited, it does not matter how big the data is. 
The value of the data is not just in the fact that they exist.
It is the knowledge, understanding, and wisdom that we gain from them that makes the data valuable.  
It is even a bigger challenge to open up the data from multiple silos.
Accessing the multi-faced data, that exist under multiple jurisdictions, about people may be prohibitively difficult.
Silos are hard to crack open.
And such data, no just Big but Deep, covering multiple facets of person's life may be invaluable for research.

Recently, we have shown how challenging but also possible is to open such institutional Big Data.
In the Data For Development (D4D) Challenge~\footnote{http://www.d4d.orange.com/home}, the telecom operator Orange opened the access to a large dataset of call detail records (CDRs) from Ivory Coast.
Working with the data as a part of a challenge, teams of researchers came up with life-changing insights for the country. 
The privacy of the people was protected not only by the technical means, such as removal of the Personally Identifiable Information (PIIs), but also by the legal means, with the researchers signing the agreement that they will not use the data for evil.
As we have seen in several cases, such as Netflix Prize privacy disaster~\cite{narayanan2008robust} and other similar privacy breaches~\cite{sweeney2000simple}, true anonymization is extremely hard and some of the weight of the privacy must rest on the legal framework.

Opening the data from the silos, by publishing static datasets is an important, but just a first step. 
We can do even more important things when the data is available in real time and can become a part of a nervous system of a society.
Epidemics and traffic congestions can be monitored and prevented in real time, underpferoming students can be helped, people with health risks can be treated before they get sick.
And the same data can be used for stalking, burglarizing one's home, and as a reason to charge people more for an insurance.

In the Unique in the Crowd project~\cite{de2013unique}, we have shown that even though human beings are highly predictable~\cite{song2010limits}, we are also very unique.
Having access to one dataset, it is easy to uniquely fingerprint someone based on just few datapoints, and use this fingerprint to discover their true identity. 
The higher the resolution of the data, the better the data, the easier it gets.

The question of privacy in this context effectively becomes the question of control. 
Who can release the data of one's movements?
To whom? 
How much and how often?
The data are collected by the institution.
The data are about people and do not belong to them, they may not even be aware that they exist.
People cannot decide upon them, cannot check them out.
People cannot delete them.
And very few can use the data, even if people wanted them to.

It does not have to be like that though.
Within the existing legal frameworks, it is possible to change the vantage point of the data ownership and put the user, the entity about whom the data are, in control.
It may be the copy of the data living in the great silo, that is being given to the user.
The user becoming the owner of their copy of the data, or whenever possible the original, owner in the old Common Law sense: with the right to use, transfer, and delete. 
An example of such mechanism in the institutional context is Blue Button initiative~\footnote{http://www.healthit.gov/bluebutton}, where the patients can get a copy of their health records.
Once the copy is with them, they can do with it as they wish: give it to someone, make it public, do research on it, destroy.

The user can accumulate data about herself from multiple sources. 
Healthcare records, mobility patterns, favorite movies, all this information belongs to the user and can be accessed based on her authorization.
This changes how and what data that can be obtained for the purpose of research and providing services.
Rather than gaining access to the movements of millions of people from a telcom operator, one can potentially gain access to a smaller number but of much richer datasets describing the users from the mobility, health, shopping  perspectives.
New startups do not have to build the user profile from scratch, but can jump in offering competitive services based on the user's collected data.
Users can immediately get better services, using their data in new places.

The first, operational challenge of moving towards the end-user data ownership on a large scale, is to create an ecosystem where such user-owned data are noticed and accessed.
We are currently embedded in a feudal framework: Facebook owns the data generated by and about their users, and provides the access to them to the 3rd parties that user might or might have not authorized. 
It is reasonably easy for users to download all their data from Facebook. 
It is reasonably easy to put it on Dropbox or even create myself-API, becoming a self-hosted API to one's own personal data. 
The challenge is to have clients to talk to this API and provide services, rather than going to Facebook for one's data. 
Today, virtually no-one is ready to access user data directly from the user. 
We have done a slightly better on the Internet scale with identity: one can deploy own OpenID server fairly easily, and many services will allow the user to sign in. We should be heading in the same direction with data.

The way the user grants the authorizations to the data she owns, is not a trivial matter.
Just answering a very simple question today 'Who is authorized to know what city I am in today' may be a challenge.
The 'Yes' user has clicked so many times, gave access to the location data to so many services.
Every tweet, every geo-tagged picture, every checkin provide one's location not only to the primary service, but also to all the applications that have been authorized to access these data.

Increasing the amount of data the user controls and increasing the granularity of the control, is meaningless if this control cannot be exercised in an informed way.
The EULA-catastrophe, where you may be just as well giving up your soul when signing up without reading, will not bring us closer to the New Deal on Data.
In the end, it must the be user that makes the informed decision about who will get the access to the data and for what purpose. 
Make the authorization interface too complex and you will fail. 
Make it too simple, and you will also fail, as you will not convey the complexity of privacy-related decisions. 
Write it in legal complex language, and you cannot claim that the user expresses informed consent. 
Start asking the user for authorization every 5 minutes, and you will only teach her to press 'Yes' every time a pop-up is presented. 

In addition to the data ownership, we need a better way for the end-user to control what happens with them. 
Will user realize that clicking this single 'Yes' provides a service with a second-resolution location data? 
And what can be inferred from such data, regarding alcohol abuse (we see you a lot in a liquor store), driving habits, not enough exercise. 
This gap between the interface and the effect, can render the data ownership meaningless. 
There is a need for Living Informed Consent, where the interface for the user to grant the authorizations is created to give the user understanding of the consequences of the granted authorizations.
This understanding will never be perfect, but aligning this user's understanding with the reality is the goal of the Living Informed Consent concept. 

We envision several ways the Living Informed Consent can improve user's understanding of the authorization she is granting. 
The underlaying principle is that the status of the authorizations expressed via the interface (website, application) is the contract.
By pressing the buttons, the user initiates technical actions (for example creation of OAuth2 tokens), but also changes her business and legal relation with the service.
Such single screen, with a timestamped log constitutes a history of the consent. 
The granularity of offered control may differ, and some actions may or may not be permitted.
Still, at any point in time, the user is in certain relation with the service, in the Business, Legal, and Technical domains. 
The consent only makes sense when the user understands what she is consenting to.
Why even bother asking otherwise.
Part of the gestalt is to provide concise authorizations description written in plain English.
They will not always be trivial and may sometime turn into paragraphs of text, still the goal should be to provide a description easily understandable for the target audience.
Additionally, the goal should be not only to ask for the access to data, but also include the purpose of the access.
Location is a type of data. Using location to provide personalized music and using location to increase my insurance for careless driving are two very different authorizations.
Currently, the widely used OAuth2 framework does not support the notion of purpose, focusing only on the data being accessed.

NOTE: expand this one

One possibility to make it easier for the user to understand what is happening with her data, is to reduce the dimensionality of the data already in the user-controlled domain, and only send high-level answers to the service requesting them.
A lot can be inferred from a raw location trace, this fact is foundational for the concept of Big Data.
The moment the raw data leaves user-controlled domain, it can be used for many things, some of them the user may have never thought about, and could not possibly have expressed informed consent.
Extracting the high-level features of the data on the user side, as described in the openPDS framework, should allow for more informed decisions regarding the data access.
All the raw data should not run wildly with every service providing a minuscule service to the user, exactly because a lot can be done with these data.
It is much easier to control what can happen and thus what are the consequences of disclosing the city you live in versus all your location updates from the last year.
It is not a perfect solution; even low-dimensionality data can still be used for evil and can be correlated with different sources.
It is however a big step in the right direction, for the user to decide upon disclosing how much liqueur she buys per week versus this information being inferred from the GPS trace provided to a service in exchange for personalized music. 

In addition, the information about data access and usage needs to be an integral part of Living Informed Consent.
How often do services sample my location?
Are they tracking me in real-time, or do they access the data on weekly basis? 
Am I singled out by them in how much they query about me, or is it the same for all the users?
Being able for the user to answers those questions in a simple, even casual way, is a crucial part for the user to remain in the state of Living Informed Consent.
Authorizations should not be of 'fire and forget' kind, instead they should be re-evaluated in some orderly fashion.
How often depends on many factors, including the sensitivity of the data accessed, reputation of the service, user preferences, the balance between control and annoyance. 

Giving the data ownership to the end-user makes it easier for the institutions to facilitate the data use.
As the user is the fully-empowered party to make decisions about authorizing access to all here data, multiple silos do not have to be visited and contracts between them made.
It is sufficient to talk to the end-user to gain access to all the data about her. 

A crucial component for realizing this vision is the identity.
It must be possible for the multiple institutions to find the user to give the data to. It must be possible for the user to identify multiple institutions and see where the data is coming from. 
All the questions that we have asked so far, who owns, who controls, who decides, who accesses, must have the 'who' component addressed.

Just as the data of the user should live under single control of the user, the identity should also be brought closer to user control.
It does not necessary mean that every user should be their identity provider, but rather than having hundreds of accounts in multiple services that do not interoperate, the identity of the user should be build on the principle of federated identity, where the services allow the user to choose their identity provider.
In addition, just like data, certain attributes of identity need to be protected.
Service does not need to know my email address to be able to log me in, a pseudonym is sufficient.
If such service has a valid reason for asking for my address, it should be based upon my grant of authorization.
Authorization that can be revoked and monitored.

In the existing system it is often hard to introduce the user data ownership. It may be for technical reasons, building infrastructure providing the space for the data.
It may be for business or legal reasons where the data is considered not suitable for sharing.
It may be for the lack of a clear incentive why to do it, and how to interact with the users.
We feel that the first step in introducing more privacy into such system, is the notion that the user must be entitled to know at least about the existence of the data about her.
The right to know about the data existence is hard to deny. It can be realized in an easier way and with less friction than transfer of the actual data. It can be the first step, enforced by legal framework.
